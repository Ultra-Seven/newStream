\section{Related Work}

Prefetching architectures for interactive data exploration are commonplace.
For instance, Chan et al.~\cite{chan2008maintaining} use a momentum-based prediction technique for time-series visualization interactions (e.g., pan, scroll, zoom) to pre-issue queries to a backend server that manages the level of detail to return to the client.  These three components---prediction, prefetching, and level of detail---is present in many modern interactive visualization system architectures.

At its core, \sys is similar in spirit, however it uses a model-based analysis to combine innovations in each component so that the end-to-end interactivity can be maintained for rich interactive interfaces and network delays.  Specifically, its prediction model is based on continuous mouse cursor movements rather than discrete database queries; its prefetching model explicitly models a distribution of user interactions; its level of detail management interleaves progressively encoded results to ensure a visualization for any future query.

Prefetching: The Web~\cite{domenech2006web,nanopoulos2003data}
Databases have studied prefetching as part of client server~\cite{ramachandran2005dynamic,ramachandraholistic,sapia2000promise,smith1978sequentiality}, 
query execution for operators such as hash-join~\cite{chen2007improving}.
It has also been specifically studied in the context of data-driven visualization~\cite{jayachandran2014combining,battle2013scalar,battle2016dynamic,cetintemel2013query,debrabant2015seer}.

Predictive Models: mouse prediction~\cite{pasqual2014mouse,lane2005process,wobbrock2009angle,wobbrock2007gestures} has been well studied.
The web search community has a notion of query intent models~\cite{li2008learning} and has used mouse movement as a way for predicting which links a user will likely click on~\cite{guo2008exploring}.
The idea of a query intent model has been explored in databases in the context of shared scan processing~\cite{ebenstein2016fluxquery}.

Progressive Encoding: multi-resolution, wavelets, online aggregation.
Saliency models have potential.  We focus on simple cases to show the value.

The distributions we use are inspired by {\it Query Intent} introduced by Nandi et al.~\cite{ebenstein2016fluxquery,gesturequery} Their work focused on shared scan optimizations for executing distributions and predicting the user's intended query specification, while we extend their use to prefetching-based execution.

