
\documentclass{vldb}
\usepackage{balance}  
\usepackage{graphicx} 
\usepackage[hyphens]{url}      
\usepackage{amsmath}
\usepackage{color}
\usepackage{cancel}
\usepackage{listings}
\usepackage[normalem]{ulem}
\usepackage{graphicx}
\usepackage{subcaption}
\setlength{\abovecaptionskip}{10pt plus 3pt minus 2pt}
\usepackage{booktabs}
\usepackage{graphicx}
\usepackage[ruled,vlined,algonl,boxed]{algorithm2e}
\usepackage{algorithmic}
\usepackage{wrapfig}
\usepackage{enumitem}
\usepackage{xspace}
\usepackage{xcolor}
\usepackage{colortbl}
\usepackage[T1]{fontenc}
\usepackage{fancyvrb}
\usepackage[bottom]{footmisc}



\usepackage[nocompress]{cite}
\usepackage{microtype}
\usepackage[section]{placeins}

\makeatletter
\def\url@leostyle{
  \@ifundefined{selectfont}{\def\UrlFont{\sf}}{\def\UrlFont{\small\bf\ttfamily}}}
\makeatother
\urlstyle{leo}

\makeatletter
\def\@copyrightspace{\relax}
\makeatother

\def\pprw{8.5in}
\def\pprh{11in}
\special{papersize=\pprw,\pprh}
\setlength{\paperwidth}{\pprw}
\setlength{\paperheight}{\pprh}
\setlength{\pdfpagewidth}{\pprw}
\setlength{\pdfpageheight}{\pprh}

\newtheorem{definition}{Definition}
\newtheorem{proposition}[definition]{Proposition}
\newtheorem{lemma}[definition]{Lemma}
\newtheorem{remark}[definition]{Remark}
\newtheorem{corollary}[definition]{Corollary}
\newtheorem{claim}[definition]{Claim}
\newtheorem{theorem}[definition]{Theorem}
\newtheorem{heuristic}[definition]{Heuristic}
\newtheorem{example}[definition]{Example}
\newtheorem{dimension}{Dimension}
\newcounter{prob}
\newtheorem{problem}[prob]{Problem}
\newtheorem{conjecture}[definition]{Conjecture}
\newtheorem{reduction}[definition]{Reduction}
\newtheorem{property}[definition]{Property}
\newtheorem{axiom}[definition]{Axiom}











\usepackage[pdftex]{hyperref}
\hypersetup{
  colorlinks=false,
  linkcolor=darkred,
  citecolor=darkgreen,
  urlcolor=darkblue
}



\widowpenalty=10000
\clubpenalty=10000


\definecolor{light-gray}{gray}{0.95}
\definecolor{mid-gray}{gray}{0.85}
\definecolor{darkred}{rgb}{0.7,0.25,0.25}
\definecolor{darkgreen}{rgb}{0.15,0.55,0.15}
\definecolor{darkblue}{rgb}{0.1,0.1,0.5}
\definecolor{blue}{rgb}{0.19,0.58,1}

\newcommand{\red}[1]{\textcolor{red}{#1}}
\newcommand{\green}[1]{\textcolor{green}{#1}}
\newcommand{\blue}[1]{\textcolor{blue}{#1}}
\newcommand{\orange}[1]{\textcolor{orange}{#1}}
\newcommand{\darkred}[1]{\textcolor{darkred}{#1}}
\newcommand{\darkgreen}[1]{\textcolor{darkgreen}{#1}}
\newcommand{\darkblue}[1]{\textcolor{darkblue}{#1}}

\makeatletter
\setlength{\@fptop}{0pt}
\makeatother



\widowpenalty 10000
\clubpenalty 10000


\newcommand{\ind}{\hspace{\algorithmicindent}}

\newcommand{\deprecate}[1]{\noindent{\color{light-gray}{#1}}}



\pagenumbering{arabic}

\makeatletter
\def\maketag@@@#1{\hbox{\m@th\normalfont\normalsize#1}}
\DeclareRobustCommand*\textsubscript[1]{
          \@textsubscript{\selectfont#1}}
        \def\@textsubscript#1{
          {\m@th\ensuremath{_{\mbox{\fontsize\sf@size\z@#1}}}}}
\makeatother

\newcommand{\papertext}[1]{#1}
\newcommand{\techreport}[1]{#1}


\newcommand{\alex}[1]{\noindent{\color{darkgreen}{[Alexandra: #1]}}}
\newcommand{\xlw}[1]{\noindent{\color{blue}{Xiaolan: #1}}}
\newcommand{\ewu}[1]{\noindent{\color{red}{EWu: #1}}}

\newcommand{\xxx}[1]{{\fontsize{13pt}{13pt}\selectfont\textcolor{red}{#1}}}
\newcommand{\codesize}{\fontsize{7}{8}}
\newcommand{\stitle}[1]{\vspace{0.5em}\noindent\textbf{#1}}
\newcommand{\calF}[0]{$\cal{F}$}

\newcommand{\sys}{VISTREAM\xspace}
\newcommand{\heurstic}{\textsc{HEURISTIC}\xspace}

\setlength\floatsep{0.8\baselineskip plus 3pt minus 2pt}
\setlength\textfloatsep{0.9\baselineskip plus 3pt minus 2pt}
\setlength\intextsep{1\baselineskip plus 3pt minus 2 pt}


\begin{document}

\title{\sys: A Prefetching Viz System That Works}

 
% \numberofauthors{1}
%  \author{
%   \alignauthor Eugene Wu\\
%     \affaddr{Computer Science}\\
%     \affaddr{Columbia University}\\
%     \email{ewu@cs.columbia.edu}\\
% }




\maketitle

\begin{abstract}
    \looseness -1
\ewu{The text is unedited for style.  It focuses on argument structure:}

It is important to make visual data exploration interactive, meaning that the system visually responds to user inputs within $100$ milliseconds.
Although the db and viz communities are seeking to do this through faster query, networking and rendering techniques,
prediction has been argued as a promising approach---if we are able to predict what the user will request, then the results can be pre-computed so that they are ready.
This can speed up user exploration as well as the visualization's initial loading time---the ``loading dataset'' progress bar is commonplace.
Such approaches rely on accurate predictive models, and existing approaches tend to limit the set of allowed user operations (e.g., to panning and zooming operations) so that the prediction space is small enough that high quality models are possible.  
Thus, the number of allowed interactions is typically significantly smaller (e.g., 1-5) as compared to the allowed interactions in real visualization systems.
Unfortunately, it's unlikely that such models will work if allow dozens of possible interactions.

We use a simple model to understand the design space and find \xxx{something interesting including a design sweet spot}.
Based on the model we design a new system, \sys, that exploits this sweet spot by using three tricks:
1) high quality prediction independent of the number of allowed operations.
2) a streaming oriented system where, in contrast to a request-response model of communication, the client periodically sends distributions
of future operations based on our high quality predictor, and the server continuously sends a stream of data to the client's circular buffer.
3) explicitly leverage progressive encodings to \xxx{something smart}.
In our experiments on XXX, we find that the unique combination of all three tricks are necessary, and
can improve responsiveness by XXX at the $95^{th}$ percentile, and YYY maximum response time.
\end{abstract}

%!TEX root = ../main.tex

\section{Introduction}

\label{s:intro}


% In spite of the growing importance of big data, sensors, and automated data collection, manual data entry continues to be a primary source of high-value data across organizations of all sizes, industries, and applications: sales representatives manage lead and sales data through SaaS applications~\cite{salesforce}; human resources, accounting, and finance departments manage employee and corporate information through terminal or internal browser-based applications~\cite{sap}; driver data is updated and managed by representatives throughout local DMV departments~\cite{dmv,dmvsystem}; consumer banking and investment data is managed through web or mobile-based applications~\cite{betterment,chase}. In all of these examples, the database is updated by translating form-based human inputs into INSERT, DELETE or UPDATE query parameters that run over the backend database---in essence, these are instances of OLTP applications that translate human input into stored procedure parameters. Unfortunately, numerous studies~\cite{kandel2012,krishnan2016hilda,Barchard20111834}, reports~\cite{citibank,Yates10,Grady13,Robeznieks05} and citizen journalists~\cite{iquantnyc} have consistently found evidence that human-generated data is both error-prone, and can significantly corrupt downstream data analyses~\cite{iquantnycnypd}. Thus, even if systems assume that data import pipelines are error-free, queries of human-driven applications continue to be a significant source of data errors, and there is a pressing need for solutions to diagnose and repair these errors. Consider the following representative toy example that we will use throughout this paper:

In this paper, we present \sys, which does some stuff.   

\begin{itemize}[leftmargin=*, topsep=0mm, itemsep=0mm]

\item We formally analyze the design space for prefetching-based visual data exploration systems to understand the conditions for which prefetching will be effective.

\item We design a novel client-server architecture.

\item We present a suite of optimizations that further reduce interaction response times, and show how to extend to new offline data structures, including sampling and data cubes.

\item We perform a thorough evaluation of the XXX characteristics \sys's performance.  

\end{itemize}

\section{Related Work}

Prefetching architectures for interactive data exploration are commonplace.
For instance, Chan et al.~\cite{chan2008maintaining} use a momentum-based prediction technique for time-series visualization interactions (e.g., pan, scroll, zoom) to pre-issue queries to a backend server that manages the level of detail to return to the client.  These three components---prediction, prefetching, and level of detail---is present in many modern interactive visualization system architectures.

At its core, \sys is similar in spirit, however it uses a model-based analysis to combine innovations in each component so that the end-to-end interactivity can be maintained for rich interactive interfaces and network delays.  Specifically, its prediction model is based on continuous mouse cursor movements rather than discrete database queries; its prefetching model explicitly models a distribution of user interactions; its level of detail management interleaves progressively encoded results to ensure a visualization for any future query.

Prefetching: The Web~\cite{domenech2006web,nanopoulos2003data}
Databases have studied prefetching as part of client server~\cite{ramachandran2005dynamic,ramachandraholistic,sapia2000promise,smith1978sequentiality}, 
query execution for operators such as hash-join~\cite{chen2007improving}.
It has also been specifically studied in the context of data-driven visualization~\cite{jayachandran2014combining,battle2013scalar,battle2016dynamic,cetintemel2013query,debrabant2015seer}.

Predictive Models: mouse prediction~\cite{pasqual2014mouse,lane2005process,wobbrock2009angle,wobbrock2007gestures} has been well studied.
The web search community has a notion of query intent models~\cite{li2008learning} and has used mouse movement as a way for predicting which links a user will likely click on~\cite{guo2008exploring}.
The idea of a query intent model has been explored in databases in the context of shared scan processing~\cite{ebenstein2016fluxquery}.

Progressive Encoding: multi-resolution, wavelets, online aggregation.
Saliency models have potential.  We focus on simple cases to show the value.

The distributions we use are inspired by {\it Query Intent} introduced by Nandi et al.~\cite{ebenstein2016fluxquery,gesturequery} Their work focused on shared scan optimizations for executing distributions and predicting the user's intended query specification, while we extend their use to prefetching-based execution.



\section{Model-based Analysis}

As described in the introduction, prefetching has the potential to enable visualization systems to respond with interactive latencies.  However, current approaches are designed for interfaces with limited options for user interaction~\cite{}; many have suggested that increasing the expressivity of the interface (e.g., the number of possible user actions) makes the prediction task significantly more challenging.  In this section, we seek to understand the minimum accuracy for model needs in order to support low user perceived latencies, as well as the conditions that affect this accuracy.

\subsection{The Model}

Our goal is to estimate the minimum prediction accuracy $\alpha$ needed in order to ensure that the user perceived latency $l_{user}$ is below a fixed threshold (e.g. 100ms~\cite{} or 500ms).
Let $T=0$ be the time.
We assume that the user will perform her request at $T=t$ where $t \ge 0$, and that the cost of answering a request consists of fetching and rendering the results.
$l_{user}$ is defined as the time between $t$ and the change reflected in the visualization. 
Let $l_{net}$ be the latency to execute a request, transfer the results across the network, and render it on screen.   In a typical system, $l_{user} = l_{net}$, which can be undesirable when query processing or the network latency are high.
Prefetching allows the client to proactively answer a request at $T=0$ and store the results in a client cache.
If a future request accessed data in the cache, then it can take much less time $l_{cache} < l_{net}$ and result in a more responsive user perceived experience.

We now model the expected user perceived latency as $l_{cache} + max(0, l_{net} - t)$ if model accurately predicted the request, and $l_{net}$ if it mispredicted.  The $max()$ operation accounts for cases when the cost of a cache miss is larger than $t$:

$$l_{user} = (l_{cache} + max(0, l_{net}-t))\times \alpha + l_{net}\times(1-\alpha) $$

Rearranging the terms, we can derive the minimum prediction accuracy in order to maintain $l_{user}$.

%$$\alpha = \frac{l_{net} - l_{user}}{l_{net} - l_{cache}}$$
$$\alpha = \frac{l_{user} - l_{net}}{l_{cache} + max(0, l_{net}-t) - l_{net}}$$

\ewu{Describe common constants} 
Figure~\ref{fig:model_base} plots the model accuracy for two commonly cited $l_{user}$ thresholds (100 and 500ms).  The x-axis varies $l_{net}$ and each line represents the percentage of the prefetching costs that the user will experience.  For instance, when $\frac{t}{l_{net}}=1$, the user initiates the request at $T=t=l_{net}$ and does not experience any of the prefetching costs, whereas when $\frac{t}{l_{net}}=0.5$, the prefetching request is only half complete by the time the user initiates her request at $T=0.5\times l_{net}$.
\ewu{Describe the figure}

\begin{figure}[ht]
	\centering
	\includegraphics[width=1\columnwidth]{figures/model_base}
 	\caption{Minimum $\alpha$ vs network latency (x-axis), $\frac{t}{l_{net}}$ ratios (lines), and two thresholds (facets).}
    \label{fig:model_base}
\end{figure}



\stitle{Vary Prefetch Concurrency}
Most modern data processing systems are able to execute multiple concurrent requests~\cite{ebenstein2016fluxquery,giannikis2012shareddb}.  
We now vary the number of concurrent prefetch requests $N$;
the $(1-\alpha)^N$ term is the probability that none of the prefetch requests match the user's actual request at $T=t$:

$$l_{user} = (l_{cache} + max(0, l_{net} - t)\times (1-(1-\alpha)^N) + l_{net}\times(1-\alpha)^N $$

Rearranging the terms results in the following minimum prediction accuracy:


$$\alpha = 1 - \left(\frac{l_{cache}+max(0,l_{net}-t)-l_{user}}{max(0,l_{net}-t)-l_{net}}\right)^{1/N}$$

\begin{figure}[h]
	\centering
	\includegraphics[width=1\columnwidth]{figures/model_concurrency}
 	\caption{Minimum $\alpha$ vs network latency (x-axis), concurrency (lines), and two latency thresholds (facets).}
  \label{fig:model_concurrency}
\end{figure}


Figure~\ref{fig:model_concurrency} shows that increasing the number of concurrent requests has an immediate effect on $\alpha$ ($t=l_{net}$ in these plots).    With $N=20$, a prediction model need only be $12\%$ accurate to ensure an interactive latency of $l_{user}=100$, while ensuring  $l_{user}=500$ only requires an accuracy of $3.5\%$. \ewu{Describe implications}



\stitle{Vary Perceived Latency Threshold}

\stitle{Network Latency Variance}

\stitle{Progressive Loading}
Assuming a tile is XXX kilobytes, and a throughput of $T mb/s$, then can sustain a concurrency level of XXX.  
When combined with progressive loading of $20\%$ (base this number off sampling and immens arguments), increases the concurrency level to XXX, 
and the model accuracy to YYY.  In many existing settings, where the number of interaction options is limited, this effectively means the prediction model
can draw randomly and be effective.  

However, in interfaces such as XXX, the number of possible interactions are roughly YYY, for which existing techniques would fail.  Under our analysis, the model simply needs to be.


\begin{figure}
	\centering
	\includegraphics[width=1\columnwidth]{figures/model_partial}
 	\caption{Required prediction accuracy as a function of network latency, under progressive conditions where partial responses are sufficient.}
    \label{fig:model_partial}
\end{figure}




\subsection{Summary of Findings}


\section{Architecture}\label{s:arch}


\begin{figure}[ht]
	\centering
	\includegraphics[width=1\columnwidth]{figures/arch}
 	\caption{System Architecture}
  \label{fig:arch}
\end{figure}

\ewu{copied from assignment readme}
Figure~\ref{fig:arch} represents the overall architecture of the system and will be used to introduce the major files.
From left to write, the developer creates visualization charts as wrappers around query templates. When the viz are instantiated, the query templates are registered with the server, so the server can instantiate the appropriate precomputed data structures. The query templates basically parameterize SQL queries.

When a user interacts with the viz (in this case, by hovering over bars), it generates parameter values for the visualization's corresponding query template. The combination is called a query, which is registered with the client Engine. If the data to answer the query is already in the client's cache, then the appropriate client data structure will compute the result and trigger a call back to update the appropriate visualizations.

If data is not available, then the requester will turn the request into a query distribution (where the query has probability 1, and everything else has probability 0) and sends the distribution to the server. The requester's other job is to continuously predict the future query distribution and send it to the server. The impulse distribution is only used when the user actually runs a query that is not in the cache.

The server's manager is basically an infinite loop that can continuously send data to the client. The instantiated data structures read from pre-computed cache files to answer queries in the current query distribution and sends a byte stream to the client.

The client recieves and caches the byte stream into a simple ring buffer. Everytime it detects a usable block of bytes, it sends it to the appropriate data structure to decode and use to answer client queries.


Data structures represent different typs of storage engines that could answer user queries. Data structure objects are represented as triangles in the architecture, they read cached files generated ahead of time (offline). They are the workhorse for answering queries.

There are a few key concepts that are important to understand how they work.

\begin{enumerate}
\item Data structures are instantiated with a query template (e.g., SELECT * WHERE a = ? AND b = ? ...) and precomputes all possible assignments to those parameters.
\item For a given data structure, it's query input is the ID of the query template and the values for the parameters.
\item Data structures send a block of byte-encoded data to the client, which needs to decode the bytes.
\item On the client, once the bytes are (optionally) decoded into javascript objects, the data structure still needs to know which queries the data can answer. To do so, we basically hash the parameter values and use it as a lookup key.
\item All of this means that we need to carefully ensure that data and queries are represented in exactly the same way in the client, the web server, and the offline scripts.
\end{enumerate}

\section{Implementation}

Our current prototype is implemented as a javascript client library and a Python server because this is common.  Performance-critical parts of the client are written in asm.js and AOT compiled.
In this section, we describe our implementation of the major components.


\subsection{Prediction Model}

Describe the challenges: prediction in general.  Note that the needed accuracy is low.  Note that KTM and mouse prediction is pretty good -- particularly if the "active" region of the interface is well established.

\ewu{Are mouse movements using interactive visualizations different from normal web browsing behavior?   Does training on one do well on the other?  vice versa? }


\stitle{Representing Distributions}
Distributions can represent thousands of possible queries.  Simply representing it efficiently is an issue.

\subsection{Storage Engines}

Data cube.

Query Template.

Sampling???  This would require client side execution engine...


\subsection{Scheduler}

The key challenge for the scheduler is to determine the number of bytes to allocate to each query in the predictied distribution, and the order to send them back.  We decompose this into two policy decisions: how to transform the distribution to account for previously sent data, and how to allocate bandwidth given a single distribution.

\stitle{Transformation:}  One approach is to schedule query results for each distribution independently and ignore past sent data.  A more advanced policy is to adjust the distributions to dampen the probability for queries that have been recently retrieved.  We call these \texttt{Indep} and \texttt{Dampen}, respectively.

\stitle{Allocation:} The second consideration is the number of bytes to allocate each We experiment with three policies: topk, proportional, Microsoft.
Top K sends all bytes for each query from highest to lowest probability.
Proportional allocates the bandwidth proportionally to all queries with a probability above a threshold $thresh$.  We use $thresh=0$ in the experiments.
Finally, Microsoft is an adaptation of a resource allocation algorithm from~\cite{} that \ewu{XXX}. 


\ewu{Note that the ring buffer does not mean data will be constantly overwritten---the scheduler can decide not to send any data if it doesn't want to overwrite something.  }

\subsection{Progressive Encoding}

We experiment with two progressive encoding schemes.  The first is haar wavelet encoding~\cite{}, and the second is running fast fourier transform~\cite{nussbaumer2012fast} (FFT) on the query results and progressively sending each component.




%!TEX root = ../main.tex


\section{Experiments}
\label{sec:experiments}

We run a series of micro and end-to-end system experiments to understand the performance improvements that each component contributes, as well as a user study to measure user-perceived latency.  These experiments use three different interactive visualization front-end designs that vary in the number of possible interactions that the user can perform---a simple pan-and-zoom map interface~\cite{}, a complex cross-filtering application, and an interface that is button-heavy.



\subsection{Experimental Setup}
This subsection describes our experimental setup and metrics.

\stitle{Traces: }
In our system experiments, we use \ewu{XXX} user traces collected from a Chromium extension running on the authors' browsers.  These traces were collected over the span of \ewu{XXX} weeks for all webpages that the authors browser.  The trace tracks mouse events (e.g., click, move, drag), as well as the type of page element that the user interacted with. 

We also collected a specialized set of user traces when interacting with custom interactive visualizations used in the user study, and labeled the type of interaction (e.g., button click, slider drag, pan, zoom in, etc).  For the visualization-based traces, we also logged the corresponding query requests for each mouse event in order to collect the ground truth.  The traces, visualizations, and queries will be released after publication.

To summarize, our traces conform to the following schemas:
{\small \begin{verbatim}
 events(eid, user, time, x, y, action, url, label)
queries(qid, eid, querystring)
\end{verbatim}}

\stitle{Conditions: } 
We use a request-response baseline ($Base_{acc}^{c}$) that performs query pre-fetching $400$ms into the future using a query prediction model with accuracy $acc$ and a FIFO cache size that can store $c$ query results.   This allows our baseline to reproduce prediction and caching characteristics from prior prefetching-based papers in a general manner.  We evaluate \sys by varying the accuracy of the mouse prediction model, the scheduling parameters, and the type of progressive encoding.  

\stitle{Metrics: }
In addition to reporting standard mouse prediction accuracy on the user traces, we report metrics for visualization quality and performance:

\begin{itemize}[leftmargin=*, topsep=0mm, itemsep=0mm]
  \item {\it Visualization Metrics: } Since \sys quickly renders visualizations that progressively improve over time, we introduce two ways to measure how accurate the visualization is.  {\it Value Error ($\epsilon_v$)} compares the difference between each mark's pixel coordinates in the progressive visualization against their coordinates in the final visualization.  For instance, if the results are rendered as a scatterplot, then we compare each point's x and y coordinates; if rendered as a bar chart, we compare each bar's pixel height.  {\it Pixel Error ($\epsilon_p$)} follows the procedure in M4~\cite{m4} and measures the number of pixels that differ in value between the progressive and final visualization.  For each measure, we report the median and $\pm 1\sigma$ bounds across time.  In addition, during our user study, we report the percentage of user interactions that achieve different $\epsilon_v$ bounds.

  \item {\it Performance Metrics: }  We report latency from user interaction to first visualization ($l_{1st}$), as well as latency until $\epsilon_v$ is below XXX ($l_{\epsilon_v}$).  
\end{itemize}

\subsection{Microbenchmarks}

This set of experiments highlight the characteristics of the mouse prediction and network schedulers in isolation.

\subsubsection{Mouse-based predictive models}

\subsubsection{Varying communication throughput and latency}
\label{sec:experiments:inc}

Vary: network throughput and latency (via client/server-side artificial delays)

\subsubsection{Throughput}

\subsubsection{Latency}


\subsection{Macrobenchmarks}

\subsection{User-Study}




%!TEX root = ../main.tex


\section{Summary and discussion}




\balance
{
\footnotesize
\bibliographystyle{abbrv}
\bibliography{main}
}

mtechreport{
\appendix

\section{Something}
\label{sec:heuristic}}






\end{document}
